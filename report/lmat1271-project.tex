\documentclass[10pt, a4paper, nofootinbib]{scrartcl}

\newcommand{\pagetitle}{LMAT1271 : Project report} 

\usepackage{amsmath}
\usepackage{amssymb}
\usepackage{amsthm}
\usepackage[T1]{fontenc}
\usepackage[french]{babel} % traduction
\usepackage[usenames,svgnames,dvipsnames]{xcolor}
%\usepackage{bookmark} 
%\usepackage{booktabs} % beautiful tables
%\usepackage[thicklines]{cancel} % cancel terms in equation
%\usepackage{color} % colors in box (theorems,...)
%\usepackage{enumerate}
\usepackage{float}
\usepackage{framed} % box around text
\usepackage{fancyhdr} % header
%\usepackage{fancyvrb}
\usepackage{mathtools}
\usepackage{mathrsfs}
\usepackage[parfill]{parskip} % avoid bad alignment in paragraphs
\usepackage{pgfplots} % plot in tikz
%\usepackage[arrowdel]{physics} % physics helpers (BUGGY)
\usepackage{derivative}
\usepackage{braket}
%\usepackage{setspace}
\usepackage{siunitx}
\usepackage{silence} % silence useless warnings
%\usepackage{systeme} % system of equations
\usepackage{tikz}
\usepackage{tkz-euclide}
\usepackage{tikz-3dplot}
\usepackage{units} % Non-stacked fractions and better unit spacing
\usepackage{xspace} % prints a trailing space in a smart way.
\usepackage{minted} % print code 

\usemintedstyle{monokay}

\usetikzlibrary{calc,patterns,angles,quotes}
\usetikzlibrary{hobby} % random curves with tikz

\tikzset{circ/.style = {fill, circle, inner sep = 0, minimum size = 3}}
\tikzset{scirc/.style = {fill, circle, inner sep = 0, minimum size = 1.5}}
\tikzset{mstate/.style={circle, draw, blue, text=black, minimum width=0.7cm}}

% \titlespacing*{\section}{0pt}{5.5ex plus 1ex minus .2ex}{4.3ex plus .2ex}
% \titlespacing*{\subsection}{0pt}{5.5ex plus 1ex minus .2ex}{4.3ex plus .2ex}

\usepackage{enumitem}
\setlist{nosep,after=\vspace{\baselineskip}}
% bullet instead of dash for items
\AtBeginDocument{\def\labelitemi{$\bullet$}}

\pgfplotsset{compat=1.12}
\sisetup{locale = FR}

\usepackage[marginparwidth=2cm]{geometry}
\geometry{
	paper=a4paper,
	inner=2.0cm,
	outer=3.0cm,
	bindingoffset=.5cm, 
	top=3.5cm, 
	bottom=3.5cm
}

\usepackage{graphicx}
\setkeys{Gin}{width=\linewidth,totalheight=\textheight,keepaspectratio}
%\graphicspath{{figures/}}

% Default images settings
\setkeys{Gin}{width=\linewidth, totalheight=\textheight, keepaspectratio}

\definecolor{blue}{rgb}{0,0,1}
\definecolor{red}{rgb}{1,0,0}

%------------------
% Théorèmes,...
%------------------
\usepackage{thmtools}
\usepackage[framemethod=TikZ]{mdframed}

\mdfdefinestyle{mdgreenbox}{%
	skipabove=8pt,
	linewidth=2pt,
	rightline=false,
	leftline=true,
	topline=false,
	bottomline=false,
	linecolor=ForestGreen,
	backgroundcolor=ForestGreen!5,
}
\declaretheoremstyle[
	headfont=\bfseries\sffamily\color{ForestGreen!70!black},
	bodyfont=\normalfont,
	postheadspace=\newline,
	spaceabove=2pt,
	spacebelow=1pt,
	mdframed={style=mdgreenbox},
	headpunct={ --- }
]{thmgreenbox}

\mdfdefinestyle{mdblackbox}{%
	skipabove=8pt,
	linewidth=3pt,
	rightline=false,
	leftline=true,
	topline=false,
	bottomline=false,
	linecolor=black,
	backgroundcolor=RedViolet!5!gray!5,
}
\declaretheoremstyle[
	headfont=\bfseries,
	bodyfont=\normalfont\small,
	spaceabove=0pt,
	spacebelow=0pt,
	mdframed={style=mdblackbox},
	postheadspace=\newline
]{thmblackbox}

\mdfdefinestyle{mdbluebox}{%
	roundcorner=10pt,
	linewidth=1pt,
	skipabove=12pt,
	innerbottommargin=9pt,
	skipbelow=1pt,
	nobreak=true,
	linecolor=blue,
	backgroundcolor=TealBlue!5,
}
\declaretheoremstyle[
	headfont=\sffamily\bfseries\color{MidnightBlue},
	mdframed={style=mdbluebox},
	headpunct={\\[3pt]}
]{thmbluebox}

\mdfdefinestyle{mdredbox}{%
	linewidth=0.5pt,
	skipabove=12pt,
	frametitleaboveskip=5pt,
	frametitlebelowskip=0pt,
	skipbelow=2pt,
	frametitlefont=\bfseries,
	innertopmargin=4pt,
	innerbottommargin=8pt,
	nobreak=true,
	linecolor=RawSienna,
	backgroundcolor=Salmon!5,
}
\declaretheoremstyle[
	headfont=\bfseries\color{RawSienna},
	mdframed={style=mdredbox},
	headpunct={\\[3pt]}
]{thmredbox}

\theoremstyle{definition}
\declaretheorem[name=Théorème,numberwithin=section,style=thmredbox]{theorem}
\declaretheorem[name=Définition,sibling=theorem,style=thmgreenbox]{definition}
\declaretheorem[name=Proposition,sibling=theorem,style=thmbluebox]{proposition}
\declaretheorem[name=Corollaire,sibling=theorem,style=thmbluebox]{corollary}
\declaretheorem[name=Lemme,sibling=theorem,style=thmbluebox]{lemma}
\declaretheorem[name=Exemple,sibling=theorem,style=thmblackbox]{example}
\declaretheorem[name=Question,sibling=theorem,style=thmblackbox]{ques}
\declaretheorem[name=Exercice,sibling=theorem,style=thmblackbox]{exercise}
\declaretheorem[name=Remarque,sibling=theorem,style=thmgreenbox]{remark}
\declaretheorem[name=Etape,style=thmgreenbox]{step}

%---------------------------------------------------
% BUGGY PACKAGES THAT NEED TO BE LOADED AT THE END
%---------------------------------------------------

% colored hyperlink
% load hyperref at the end to avoid conflicts
\usepackage[colorlinks]{hyperref} % colored ref
% % cleverref must be loaded after hyperref
% \usepackage{cleveref} % create ref

\hypersetup{
  colorlinks=true,
  linkcolor=blue,
  filecolor=blue,
  citecolor=black,
  urlcolor=cyan
}

% conditions for equations
% https://tex.stackexchange.com/questions/95838/how-to-write-a-perfect-equation-parameters-description
\newenvironment{conditions}
  {\par\vspace{\abovedisplayskip}\noindent\begin{tabular}{>{$}l<{$} @{${}={}$} l}}
	{\end{tabular}\par\vspace{\belowdisplayskip}}
	
\newenvironment{conditions*}
  {\par\vspace{\abovedisplayskip}\noindent
   \tabularx{\columnwidth}{>{$}l<{$} @{${}={}$} >{\raggedright\arraybackslash}X}}
  {\endtabularx\par\vspace{\belowdisplayskip}}

%----------------
% FIX
%----------------

\setlength{\headheight}{14.5pt}

% increase vertical space for aligned equations
\setlength{\jot}{7pt}

% Filter warnings issued by package biblatex starting with "Patching footnotes failed"
\WarningFilter{biblatex}{Patching footnotes failed}

%----------------------
%	COMMANDS
%----------------------

% commands shortcuts
\newcommand{\mb}{\mathbb}
\newcommand{\R}{\mb{R}}
\newcommand{\Z}{\mb{Z}}
\newcommand{\N}{\mb{N}}
\newcommand{\C}{\mb{C}}
\newcommand{\A}{\mb{A}} % hypersphere area
\newcommand{\V}{\mb{V}} % hypersphere volume
\newcommand{\dS}{\cdot d\vec{S}}
\newcommand{\lag}{\mathcal{L}}
\newcommand{\ham}{\mathcal{H}}
\newcommand{\Mod}[1]{\ \mathrm{mod}\ #1}
\newcommand{\res}{\text{res}}
\newcommand{\ind}{\text{ind}}
\newcommand{\carg}{\text{arg}}

\newcommand{\vb}[1]{\mathbf{\vec{#1}}} % bold vectors
\newcommand{\vd}[1]{\dot{\vec{#1}}} 
\newcommand{\vdd}[1]{\ddot{\vec{#1}}}

% norm
\newcommand{\bignorm}[1]{\left\lVert#1\right\rVert}

% smaller overline (line above variable)
\newcommand{\overbar}[1]{\mkern 1.5mu\overline{\mkern-1.5mu#1\mkern-1.5mu}\mkern 1.5mu}

% prints an asterisk that takes up no horizontal space.
% useful in tabular environments.
\newcommand{\hangstar}{\makebox[0pt][l]{*}}

% Prints argument within hanging parentheses (i.e., parentheses that take
% up no horizontal space). Useful in tabular environments.
\newcommand{\hangp}[1]{\makebox[0pt][r]{(}#1\makebox[0pt][l]{)}}

% cancel terms with color
% \newcommand{\ccancel}[2]{\renewcommand{\CancelColor}{\color{#2}}\bcancel{#1}}

% small parallel 
\makeatletter
\newcommand{\newparallel}{\mathrel{\mathpalette\new@parallel\relax}}
\newcommand{\new@parallel}[2]{%
  \begingroup
  \sbox\z@{$#1T$}% get the height of an uppercase letter
  \resizebox{!}{\ht\z@}{\raisebox{\depth}{$\m@th#1/\mkern-5mu/$}}%
  \endgroup
}
\makeatother

\renewcommand*\contentsname{Table des matières}
\usepackage{scrlayer-scrpage}
\pagestyle{scrheadings}

\setheadsepline{0.4pt} % Ligne au-dessus de la page
\setfootsepline{0.4pt} % Ligne au-dessous de la page

% header
\setkomafont{pagehead}{\bfseries}  % police en-tête 
\lohead{\pagetitle}  
\ohead{Mathieu Rousseau}

% footer
\ifoot{2020-2021}
\ofoot{\pagemark}

\pagenumbering{arabic}

\begin{document}

\section{Point estimation}

\subsection*{Context}
Our engineering team just landed a consulting contract with a company interested in the electricity consumption of its machines. In a first part, we would like to determine how electricity consumption is evenly distributed across the different machines of the same type. To this end, we use the Gini coefficient. In a nutshell, it is an index ranging from $0$ to $1$ measuring the inequality featured in a distribution. A value of $0$ denotes that all our machines use the same amount of electricity while a value of $1$ means that all the electricity is used by a single machine.
We assume that all of the $n$ machines operate independently and their daily electricity consumption (in MWh) can be modelled as a random variable $X$ with the following density function,

\begin{equation}
  f_{\theta_1, \theta_2}(x) = 
  \begin{cases}
    \frac{\theta_1 \theta_2^{\theta_1}}{x^{\theta_1 + 1}}, &\quad x \geq \theta_2 \\
    0,                                                     &\quad \text{otherwise}
  \end{cases}
\end{equation}

with $\theta_1 > 2$ and $\theta_2 > 0$.

\textbf{(a)} Derive the quantile function of $X$

\begin{center}\rule{6cm}{0.4pt}\end{center}

We're looking to solve $P(X \leq x_t) = t$ for $x_t$.

First let's compute $P(X \leq x_t)$, 
\begin{align*}
  P(X \leq x_t)
    &= \int_{-\infty}^{x_t} f_{\theta_1, \theta_2}(x) dx \\
    &= \int_{\theta_2}^{x_t} \theta_1 \theta_2^{\theta_1} x^{-(\theta_1 + 1)} dx \\
    &= - \frac{\theta_1 \theta_2^{\theta_1}}{\theta_1} \left[ x^{-\theta_1} \right]_{x=\theta_2}^{x=x_t} \\
    &= - \frac{\theta_1 \theta_2^{\theta_1}}{\theta_1} \left( x_t^{-\theta_1} - \theta_2^{-\theta_1} \right) \\
\end{align*}

Let's solve $P(X \leq x_t) = t$ for $x_t$,
\begin{align*}
  - \frac{\theta_1 \theta_2^{\theta_1}}{\theta_1} \left( x_t^{-\theta_1} - \theta_2^{-\theta_1} \right) = t 
    &\iff x_t^{\theta_1} 
      = \frac{t \theta_1}{\theta_1 \theta_2^{\theta_1}} - \theta_2^{-\theta_1} \\
    &\iff x_t 
      = \left( \frac{t \theta_1}{\theta_1 \theta_2^{\theta_1}} - \theta_2^{-\theta_1} \right)^{1/\theta_1} \equiv Q_{\theta_1, \theta_2}(t)
\end{align*}

\textbf{(b)} Derive the Gini coefficient of $X$.

\begin{center}\rule{6cm}{0.4pt}\end{center}

The Gini coefficient is defined as, 
\begin{equation}
  G_{\theta_1, \theta_2}(t) = 2 \int_{0}^{1} \left( p - \frac{\int_{0}^{p} Q(t) dt}{E(X)} \right) dp
\end{equation}

Let's first compute the mean of $X$,
\begin{align*}
  E(X)
    &= \int_{-\infty}^{+\infty} x f(x) dx \\
    &= \int_{\theta_2}^{+\infty} x \frac{\theta_1 \theta_2^{\theta_1}}{x^{\theta_1 + 1}} dx \\
    &= \theta_1 \theta_2^{\theta_1} \int_{\theta_2}^{+\infty} x^{- \theta_1} dx \\
    &= - \frac{\theta_1 \theta_2^{\theta_1}}{(\theta_1 - 1)} \left[ x^{-(\theta_1 - 1)} \right]_{\theta_2}^{+\infty}
\end{align*}

Then the Gini coefficient, 
\begin{align*}
  G_{\theta_1, \theta_2}
    &= 2 \left( \int_{0}^{1} p dp - \int_{0}^{1} \frac{\int_{0}^{p} Q(t) dt}{E(X)} dp \right) 
\end{align*}

We compute each integral separately,
\begin{align*}
  \int_0^p Q(t) dt 
    &= \int_0^p \left( \frac{t \theta_1}{\theta_1 \theta_2^{\theta_1}} - \theta_2^{-\theta_1} \right)^{1/\theta_1} dt
\end{align*}

We use the change of variable $u = \frac{t \theta_1}{\theta_1 \theta_2^{\theta_1}} - \theta_2^{-\theta_1}$, $du = \frac{\theta_1}{\theta_1 \theta_2^{\theta_1}} dt$

The boundaries becomes, 
\begin{align*}
  \begin{cases}
    t = 0 &\implies u_1 \equiv - \theta_2^{-\theta_1} \\
    t = p &\implies u_2 \equiv \frac{p \theta_1}{\theta_1 \theta_2^{\theta_1}} - \theta_2^{-\theta_1}
  \end{cases}
\end{align*}

Then, 
\begin{align*}
  \int_0^p Q(t) dt 
    &= \int_{u_1}^{u_2} u^{(1/\theta_1)} \frac{\theta_1 \theta_2^{\theta_1}}{\theta_1} du \\
    &= \frac{\theta_1}{\theta_1 \theta_2^{\theta_1}} \left[ \frac{u^{(1/\theta_1) + 1}}{(1/\theta_1) + 1} \right]_{u_1}^{u_2} \\
    &= \frac{\theta_1}{\theta_1 \theta_2^{\theta_1} ((1/\theta_1) + 1)} \left( \left( \frac{p \theta_1}{\theta_1 \theta_2^{\theta_1}} - \theta_2^{-\theta_1} \right)^{(1/\theta_1) + 1} - \left( - \theta_2^{-\theta_1} \right)^{(1/\theta_1) + 1} \right)
\end{align*}

Therefore, 
\begin{align*}
  \frac{\int_{0}^{p} Q(t) dt}{E(X)}
    &= \frac{\theta_1(\theta_1 - 1)}{\theta_2^{(1 - \theta_1)} ((1/\theta_1) + 1)} \left( \left( \frac{p \theta_1}{\theta_1 \theta_2^{\theta_1}} - \theta_2^{-\theta_1} \right)^{(1/\theta_1) + 1} - \left( - \theta_2^{-\theta_1} \right)^{(1/\theta_1) + 1} \right)
\end{align*}

Then,
\begin{align*}
  \int_{0}^{1} \frac{\int_{0}^{p} Q(t) dt}{E(X)} dp 
    &= \frac{\theta_1(\theta_1 - 1)}{\theta_2^{1 - \theta_1}} \left( \underbrace{\int_0^1 \left( \frac{p \theta_1}{\theta_1 \theta_2^{\theta_1}} - \theta_2^{-\theta_1} \right)^{(1/\theta_1) + 1} dp}_{\equiv A} - \underbrace{\int_0^1 \left( - \theta_2^{-\theta_1} \right)^{(1/\theta_1) + 1} dp}_{\equiv B} \right)
\end{align*}

Computing integral A and B. For A we use the same change of variable as before,
\begin{align*}
  A 
    &= \int_{u_1}^{u_2} u^{(1/\theta_1) + 1} \frac{\theta_1 \theta_2^{\theta_1}}{\theta_1} du \\
    &= \frac{\theta_1 \theta_2^{\theta_1}}{\theta_1} \left( \left( \frac{\theta_1}{\theta_1 \theta_2^{\theta_1}} - \theta_2^{-\theta_1} \right)^{(1/\theta_1) + 2} - \left( - \theta_2^{-\theta_1} \right)^{(1/\theta_1) + 2} \right)
\end{align*}

\begin{align*}
  B 
    &= \left( - \theta_2^{-\theta_1} \right)^{(1/\theta_1) + 1} \int_0^1 dp \\
    &= \left( - \theta_2^{-\theta_1} \right)^{(1/\theta_1) + 1}
\end{align*}

Then, 
\begin{align*}
  \int_0^1 pdp 
    &= \frac{1}{2}
\end{align*}

Eventually,
\begin{align*}
  G_{\theta_1, \theta_2} 
    &= 2 \left( \frac{1}{2} - \frac{\theta_1(\theta_1 - 1)}{\theta_2^{(1 - \theta_1)} ((1/\theta_1) + 1)} \left[ \frac{\theta_1 \theta_2^{\theta_1}}{\theta_1} \frac{1}{(1/\theta_1) + 2} \left( \left( \frac{\theta_1}{\theta_1 \theta_2^{\theta_1}} - \theta_2^{-\theta_1} \right)^{(1/\theta_1) + 2} - \left( - \theta_2^{-\theta_1} \right)^{(1/\theta_1) + 2} \right) \\
    & - \left( - \theta_2^{-\theta_1} \right)^{(1/\theta_1) + 1} \right] \right) 
\end{align*}

\end{document}
